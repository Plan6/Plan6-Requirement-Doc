\documentclass[]{article}

% Imported Packages
%------------------------------------------------------------------------------
\usepackage{amssymb}
\usepackage{amstext}
\usepackage{amsthm}
\usepackage{amsmath}
\usepackage{enumerate}
\usepackage{fancyhdr}
\usepackage[margin=1in]{geometry}
\usepackage{graphicx}
\usepackage{extarrows}
\usepackage{setspace}
%------------------------------------------------------------------------------

% Header and Footer
%------------------------------------------------------------------------------
\pagestyle{plain}
\renewcommand\headrulewidth{0.4pt}
\renewcommand\footrulewidth{0.4pt}
%------------------------------------------------------------------------------

% Title Details
%------------------------------------------------------------------------------
\title{Deliverable \#1 Template}
\author{SE 3A04: Software Design II -- Large System Design}
\date{}
%------------------------------------------------------------------------------

% Document
%------------------------------------------------------------------------------
\begin{document}

\maketitle

\section{Introduction}
\label{sec:introduction}
% Begin Section

This section of the SRS should provide an overview of the entire SRS.

\subsection{Purpose}
\label{sub:purpose}
% Begin SubSection
\begin{enumerate}[a)]
	\item Delineate the purpose of the SRS
	\item Specify the intended audience for the SRS
\end{enumerate}
% End SubSection

\subsection{Scope}
\label{sub:scope}
% Begin SubSection
\begin{enumerate}[a)]
	\item Identify the software product(s) to be produced by name (e.g., Host DBMS, 
Report Generator, etc.)
	\item Explain what the software product(s) will, and, if necessary, will not do
	\item Describe the application of the software being specified, including 
relevant benefits, objectives, and goals
	\item Be consistent with similar statements in higher-level specifications 
(e.g., the system requirements specification), if they exist
\end{enumerate}
% End SubSection

\subsection{Definitions, Acronyms, and Abbreviations}
\label{sub:definitions_acronyms_and_abbreviations}
% Begin SubSection
\begin{enumerate}[a)]
	\item Provide the definitions of all terms, acronyms, and abbreviations 
required to properly interpret the SRS
\end{enumerate}
% End SubSection

\subsection{References}
\label{sub:references}
% Begin SubSection
\begin{enumerate}[a)]
	\item Provide a complete list of all documents referenced elsewhere in the SRS
	\item Identify each document by title, report number (if applicable), date, and 
publishing organization
	\item Specify the sources from which the references can be obtained
\end{enumerate}
% End SubSection

\subsection{Overview}
\label{sub:overview}
% Begin SubSection
\begin{enumerate}[a)]
	\item Describe what the rest of the SRS contains
	\item Explain how the SRS is organized
\end{enumerate}
% End SubSection

% End Section

\section{Overall Description}
\label{sec:overall_description}
% Begin Section

This section of the SRS should describe the general factors that affect the 
product and its requirements. It does not state specific requirements; it 
provides a background for those requirements and makes them easier to 
understand.

\subsection{Product Perspective}
\label{sub:product_perspective}
% Begin SubSection
Plan 6 is a distributed learning management system designed to facilitate 
teachers and students with everyday classroom operations. Plan 6 is a 
distributed, self-contained system with several components running on different 
end-user devices. Plan 6 is akin to other learning management systems, such as 
McMaster's Avenue2Learn; however, Plan 6 will focus on providing a Learning 
Management System for early learners and educators with low levels of technical 
expertise. 

A block diagram denoting the major components of system as well as their 
external interfaces follow:

\begin{center}
\includegraphics[scale=0.7]{A1_Assets/2-1_Product_Perspective_Diagram.jpg}
\end{center}

% End SubSection

\subsection{Product Functions}
\label{sub:product_functions}
% Begin SubSection
Edurite-LMS is designed to facilitate students, teachers, and administration 
with common classroom operations. 

As a LMS, Edurite-LMS will allow teachers to upload a variety of documents for 
the purpose of education. Teachers are then able to set the duration that the 
documents are available, as well as the the access rights users have over the 
document.

Moreover, Edurite-LMS will allow teachers to create virtual mechanisms of 
assessment including but not limited to quizzes, and electronic assignment 
submission boxes. With virtual quizzes, teachers will be allowed to create their 
own questions and solutions.

Teachers will also be able to configure settings related to these mechanisms of 
assessment including but not limited to the availability of the mechanism, 
functionalities available while users are participating in the mechanism of 
assessment, and the assessment scheme associated with the mechanism. 

The LMS will also allow teachers to enter success metrics for each student. If 
instructed to do so, the LMS will also be able to automatically assess students 
based on primitive assessment schemes provided by the teacher.

Edurite-LMS will allow teachers to view statistics associated with each 
mechanism of assessment. It will create visualizations of data generated from 
the results of the assessment scheme and display them to the teacher. Statistics 
to be displayed to the teacher include but is not limited to the statistical 
mean, the statistical mode, the variance, the standard deviation, the point 
biserial, and the discrimination index.

On the other hand, Edurite-LMS will allow students to view documents uploaded by 
the teacher. The students will be able to save a copy of the document if it is 
allowed by the teacher. 

Moreover, Edurite-LMS will allow students to complete virtual mechanisms of 
assessments created by their teacher. The LMS will allow students to submit the 
appropriate information in order to successfully complete the mechanism of 
assessment. The LMS will show the student a metric of success when authorized by 
the teacher.

Finally, Edurite-LMS will allow administration to add or remove students from 
certain domains. The LMS will allow administration to modify the user roles and 
to control what permissions each role has over the system.
% End SubSection

\subsection{User Characteristics}
\label{sub:user_characteristics}
% Begin SubSection
Edurite-LMS is intended for 3 primary groups of users: teachers, students, and 
administration.

Teachers are assumed to be adult users with the functional capacity of an 
average person who is between the ages of 16 and 80. Teachers are assumed to be 
fluent in written English, and have at least a basic understanding of 
mathematics and statistics. Moreover, teachers are assumed to have a strong 
understanding of teaching methods; however, teachers are not assumed to be 
well-versed in technology. Teachers should know how to use an Android platform 
- including typing and selecting elements on the screen - however, they are not 
expected to know how to know advanced features of the phone.

Students are assumed to be child users of ages 6 to 13, who are enrolled in a 
Canadian, American, or equivalent elementary school. Students are assumed to 
understand basic written English at the Ontario Senior Kindergarten/Grade 1 
level. Students are expected to have vision and are able to use the basic 
functions of the Android platform, albeit at a lower level than teachers and 
administrations. At a minimum, students are assumed to know how to select 
elements on the screen. No other prior knowledge is assumed of the student.

The administration is assumed to be adult users with the functional capacity of 
an average person who is between the ages of 16 to 80. The administration is 
assumed to be fluent in written English, and have a good understanding of 
mathematics and statistics. The administration is also assumed to have basic 
computer knowledge, including basic troubleshooting, IT, and data entry skills. 
The administration is assumed to have an intermediate understanding of common 
Android platform usage, and are able to preform more advanced tasks such as 
set-up, installation, and troubleshooting on such platforms. 
% End SubSection

\subsection{Constraints}
\label{sub:constraints}
% Begin SubSection
The application must run on the native Android platform, as per the project 
specifications provided.
% End SubSection

\subsection{Assumptions and Dependencies}
\label{sub:assumptions_and_dependencies}
% Begin SubSection
It is assumed that the school using the system will have the necessary infrastructure to support the system. The school will provide the required hardware in order to reasonably ensure the security of the data, and to handle the data capacity requirement.

Moreover, it is assumed that each user will have access to an Android platform that is internet enabled.
% End SubSection

\subsection{Apportioning of Requirements}
\label{sub:apportioning_of_requirements}
% Begin SubSection
This section is void.
% End SubSection

% End Section

\section{Functional Requirements}
\label{sec:functional_requirements}
% Begin Section
This section of the SRS should contain all of the software requirements to a 
level of detail sufficient to enable designers to design a system to satisfy 
those requirements, and testers to test that the system satisfies those 
requirements. Throughout this section, every stated requirement should be 
externally perceivable by users, operators, or other external systems. These 
requirements should include at a minimum a description of every input (stimulus) 
into the system, every output (response) from the system, and all functions 
performed by the system in response to an input or in support of an output.

You normally have two options for organizing your functional requirements:
\begin{enumerate}
	\item Organize first by \emph{business events}, then by \emph{viewpoints}
	\item Organize first by \emph{viewpoints}, then by \emph{business events}
\end{enumerate}
Choose the one which makes the most sense.

For example, if you wish to organization by business events:

% It is assumed that we will take this approach.
%
\begin{enumerate}[{BE}1.]
	\item A teacher must be able to create a course and own it or delete an 
existing course the teacher owns.
	\begin{enumerate}[{VP1}.1]
		\item Student Viewpoint
			\begin{enumerate}
				\item Students must be able to enroll in a course owned by a teacher.
			\end{enumerate}
		\item Teacher Viewpoint
			\begin{enumerate}
				\item A teacher may choose to unenroll a student from a course that the 
teacher owns.
				\item A teacher is not enrolled in a course.
			\end{enumerate}
		\item Parent Viewpoint
			\begin{enumerate}
				\item Should a parent wish to see their course progress or student 
performance, they are required to use the child's student account.
			\end{enumerate}
		\item School Administration Viewpoint
			\begin{enumerate}
				\item The school administration may choose to either allow students and 
teachers to create their own accounts, or create accounts for all students and 
teachers.
			\end{enumerate}
		\item Information Technology Team Viewpoint
			\begin{enumerate}
				\item None
			\end{enumerate}
		\item Government \& School Board Viewpoint
			\begin{enumerate}
				\item All courses, course content, and any teacher student communication 
held on the CMS must be made visible to the School Board.
			\end{enumerate}
	\end{enumerate}

	\item A teacher must be able to add non-evaluational, manual-evaluational and 
automated-evaluational content to a course created by that teacher.
	\begin{enumerate}[{VP1}.1]
		\item Student Viewpoint
			\begin{enumerate}
				\item A student enrolled in a course may view any content within the course 
provided the content is set visible by the owner of that course.
				\item A student may submit a piece of content in response to a 
manual-evaluational content provided the manual-evaluational content is set 
visible by the owner of that course.
				\item A student may submit an automated evaluational content provided it is 
set visible by the owner of that course.
			\end{enumerate}
		\item Teacher Viewpoint
			\begin{enumerate}
				\item Non-evaluational content is any content for which students are given 
only the option to view the content.
				\item Manual-evaluational content is any content where, in addition to 
viewing, students may post their own content in response, called a submitted 
assignment, to the content posted by the teacher.
				\item Automated-evaluational content is any content that when accessed by 
students, allows students to interact with that content and the result of their 
interaction is marked by a previously defined success criteria defined by the 
owner of the course.
				\item The student's performance in the exploration of automated-evaluational 
content is recorded and saved for future reference by the student and the 
teacher.
			\end{enumerate}
		\item Parent Viewpoint
			\begin{enumerate}
				\item None
			\end{enumerate}
		\item School Administration Viewpoint
			\begin{enumerate}
				\item None
			\end{enumerate}
		\item Information Technology Team Viewpoint
			\begin{enumerate}
				\item None
			\end{enumerate}
		\item Government \& School Board Viewpoint
			\begin{enumerate}
				\item None
			\end{enumerate}
	\end{enumerate}

	\item A teacher must be able view a submitted assignment. Then may choose to 
assign either a numerical or letter grade to the assignment.
	\begin{enumerate}[{VP2}.1]
		\item Student Viewpoint
			\begin{enumerate}
				\item Students may submit assignments multiple times in response to a 
manual-evaluational content posted by the course owner.
			\end{enumerate}
		\item Teacher Viewpoint
			\begin{enumerate}
				\item None
			\end{enumerate}
		\item Parent Viewpoint
			\begin{enumerate}
				\item None
			\end{enumerate}
		\item School Administration Viewpoint
			\begin{enumerate}
				\item None
			\end{enumerate}
		\item Information Technology Team Viewpoint
			\begin{enumerate}
				\item None
			\end{enumerate}
		\item Government \& School Board Viewpoint
			\begin{enumerate}
				\item None
			\end{enumerate}
	\end{enumerate}

	\item A teacher must be able to add a quiz or delete an existing quiz.
	\begin{enumerate}[{VP2}.1]
		\item Student Viewpoint
			\begin{enumerate}
				\item None
			\end{enumerate}
		\item Teacher Viewpoint
			\begin{enumerate}
				\item None
			\end{enumerate}
		\item Parent Viewpoint
			\begin{enumerate}
				\item None
			\end{enumerate}
		\item School Administration Viewpoint
			\begin{enumerate}
				\item None
			\end{enumerate}
		\item Information Technology Team Viewpoint
			\begin{enumerate}
				\item None
			\end{enumerate}
		\item Government \& School Board Viewpoint
			\begin{enumerate}
				\item None
			\end{enumerate}
	\end{enumerate}

	\item The owner of a course must be able to specify a time period for which a 
non-evaluational, automated-evaluational, manual-evaluational or a course is 
visible for to all enrolled students.
	\begin{enumerate}[{VP2}.1]
		\item Student Viewpoint
			\begin{enumerate}
				\item An enrolled student may not view content which is not made visible by 
the owner of the course.
			\end{enumerate}
		\item Teacher Viewpoint
			\begin{enumerate}
				\item A teacher may choose to permanently remove a piece of content from a 
course owned by them.
			\end{enumerate}
		\item Parent Viewpoint
			\begin{enumerate}
				\item None
			\end{enumerate}
		\item School Administration Viewpoint
			\begin{enumerate}
				\item None
			\end{enumerate}
		\item Information Technology Team Viewpoint
			\begin{enumerate}
				\item None
			\end{enumerate}
		\item Government \& School Board Viewpoint
			\begin{enumerate}
				\item None
			\end{enumerate}
	\end{enumerate}

	\item A student must be able to enroll within a course that is set visible by 
the owner of that course.
	\begin{enumerate}[{VP2}.1]
		\item Student Viewpoint
			\begin{enumerate}
				\item A student must be able view all courses to which they are enrolled.
				\item A student may not self withdraw from a course in which he or she is 
enrolled.
				\item A student that is dismissed from a course is not longer enrolled in 
that course.
				\item A student whom is not enrolled in a course has no access to content 
found within that course.
			\end{enumerate}
		\item Teacher Viewpoint
			\begin{enumerate}
				\item A teacher may dismiss an enrolled student from a course owned by the 
teacher.
			\end{enumerate}
		\item Parent Viewpoint
			\begin{enumerate}
				\item None
			\end{enumerate}
		\item School Administration Viewpoint
			\begin{enumerate}
				\item None
			\end{enumerate}
		\item Information Technology Team Viewpoint
			\begin{enumerate}
				\item None
			\end{enumerate}
		\item Government \& School Board Viewpoint
			\begin{enumerate}
				\item None
			\end{enumerate}
	\end{enumerate}

	\item A student must be able to view all grades assigned to them for all 
assignments and quizzes for all courses in which the student is enrolled.
	\begin{enumerate}[{VP2}.1]
		\item Student Viewpoint
			\begin{enumerate}
				\item None
			\end{enumerate}
		\item Teacher Viewpoint
			\begin{enumerate}
				\item None
			\end{enumerate}
		\item Parent Viewpoint
			\begin{enumerate}
				\item None
			\end{enumerate}
		\item School Administration Viewpoint
			\begin{enumerate}
				\item None
			\end{enumerate}
		\item Information Technology Team Viewpoint
			\begin{enumerate}
				\item None
			\end{enumerate}
		\item Government \& School Board Viewpoint
			\begin{enumerate}
				\item None
			\end{enumerate}
	\end{enumerate}

	\item A student must be able to view all assignments within a course in which 
they are enrolled provided that the owner of the course has marked them visible 
to be all students.
	\begin{enumerate}[{VP2}.1]
		\item Student Viewpoint
			\begin{enumerate}
				\item None
			\end{enumerate}
		\item Teacher Viewpoint
			\begin{enumerate}
				\item None
			\end{enumerate}
		\item Parent Viewpoint
			\begin{enumerate}
				\item None
			\end{enumerate}
		\item School Administration Viewpoint
			\begin{enumerate}
				\item None
			\end{enumerate}
		\item Information Technology Team Viewpoint
			\begin{enumerate}
				\item None
			\end{enumerate}
		\item Government \& School Board Viewpoint
			\begin{enumerate}
				\item None
			\end{enumerate}
	\end{enumerate}

	\item A student may submit a file in response to a manual-evaluational content 
only if the manual evaluational content was made visible to all students and if 
a submission was allowed for that student.
	\begin{enumerate}[{VP2}.1]
		\item Student Viewpoint
			\begin{enumerate}
				\item None
			\end{enumerate}
		\item Teacher Viewpoint
			\begin{enumerate}
				\item None
			\end{enumerate}
		\item Parent Viewpoint
			\begin{enumerate}
				\item None
			\end{enumerate}
		\item School Administration Viewpoint
			\begin{enumerate}
				\item None
			\end{enumerate}
		\item Information Technology Team Viewpoint
			\begin{enumerate}
				\item None
			\end{enumerate}
		\item Government \& School Board Viewpoint
			\begin{enumerate}
				\item None
			\end{enumerate}
	\end{enumerate}

	\item The school administration must be able to add a Teacher or a Student 
account or delete an existing Teacher or Student account.
	\begin{enumerate}[{VP1}.1]
		\item Student Viewpoint
			\begin{enumerate}
				\item The student recieves a Student ID and password to access their 
personal student account.
				\item The system must force student and
			\end{enumerate}
		\item Teacher Viewpoint
			\begin{enumerate}
				\item The teacher recieves a Teacher ID and password to access their 
personal teacher account.
			\end{enumerate}
		\item Parent Viewpoint
			\begin{enumerate}
				\item Should the parent require access to their child's account, the child 
is expected to login through their student account to allow the parent to 
inspect course progress or student performance.
			\end{enumerate}
		\item School Administration Viewpoint
			\begin{enumerate}
				\item The School Administration must be able to create batch set of user 
names and passwords from a CSV file.
				\item The passwords for the students and teachers are generated as follows 
DDMMYYYY where DDMMYYYY refers to the date of birth of the student or teacher.
			\end{enumerate}
		\item Information Technology Team Viewpoint
			\begin{enumerate}
				\item None
			\end{enumerate}
		\item Government \& School Board Viewpoint
			\begin{enumerate}
				\item It is expected that the responsibilities of monitoring communication 
is forwarded to the school administration. All enquiries are assumed to be 
forwarded to the school administration that are expected to use special access 
privileges to investigate issues regarding content or communication.
				\item Should any content maliagn against board policies, the school 
administration are given access and privileges to hide, remove or track the 
origin of a piece of content or course.
			\end{enumerate}
	\end{enumerate}

	\item The school administration must be able to view a course created by a 
Teacher or delete an existing course created by a Teacher.
	\begin{enumerate}[{VP1}.1]
		\item Student Viewpoint
			\begin{enumerate}
				\item If a course is deleted by the administration, all student data 
associated with that course is also deleted.
			\end{enumerate}
		\item Teacher Viewpoint
			\begin{enumerate}
				\item If a course is deleted by the administration, all teacher data 
associated with that course is also deleted.
			\end{enumerate}
		\item Parent Viewpoint
			\begin{enumerate}
				\item None
			\end{enumerate}
		\item School Administration Viewpoint
			\begin{enumerate}
				\item The school administration account is given a warning after a course 
deletion request is made. The course is deleted if and only if a course deletion 
request is made a second time following the warning.
			\end{enumerate}
		\item Information Technology Team Viewpoint
			\begin{enumerate}
				\item None
			\end{enumerate}
		\item Government \& School Board Viewpoint
			\begin{enumerate}
				\item It is assumed that the school administration will consult policies and 
procedures regarding the handling of student data.
			\end{enumerate}
	\end{enumerate}

	\item The school administration must be able to view all communication that 
occurs within the system between student and teachers, between student and 
school administration or between school administration and teachers.
	\begin{enumerate}[{VP1}.1]
		\item Student Viewpoint
			\begin{enumerate}
				\item All uploaded content or other forms of communication the student 
produces must be visible to all school administration accounts.
			\end{enumerate}
		\item Teacher Viewpoint
			\begin{enumerate}
				\item All uploaded content or other forms of communication the teacher 
produces must be visible to all school administration accounts.
			\end{enumerate}
		\item Parent Viewpoint
			\begin{enumerate}
				\item None
			\end{enumerate}
		\item School Administration Viewpoint
			\begin{enumerate}
				\item None
			\end{enumerate}
		\item Information Technology Team Viewpoint
			\begin{enumerate}
				\item None
			\end{enumerate}
		\item Government \& School Board Viewpoint
			\begin{enumerate}
				\item None
			\end{enumerate}
	\end{enumerate}

	\item Description of Business Event
	\begin{enumerate}[{VP1}.1]
		\item Student Viewpoint
			\begin{enumerate}
				\item None
			\end{enumerate}
		\item Teacher Viewpoint
			\begin{enumerate}
				\item None
			\end{enumerate}
		\item Parent Viewpoint
			\begin{enumerate}
				\item None
			\end{enumerate}
		\item School Administration Viewpoint
			\begin{enumerate}
				\item None
			\end{enumerate}
		\item Information Technology Team Viewpoint
			\begin{enumerate}
				\item None
			\end{enumerate}
		\item Government \& School Board Viewpoint
			\begin{enumerate}
				\item None
			\end{enumerate}
	\end{enumerate}

\end{enumerate}

% End Section

\section{Non-Functional Requirements}
\label{sec:non-functional_requirements}
% Begin Section
\subsection{Look and Feel Requirements}
\label{sub:look_and_feel_requirements}
% Begin SubSection

\subsubsection{Appearance Requirements}
\label{ssub:appearance_requirements}
% Begin SubSubSection
\begin{enumerate}[{LF}1. ]
	\item The application shall look aesthetically pleasing
	\item The application shall be simplistic enough for anyone to be able to use
	\item The application shall use standard fonts
	\item The application shall not have clashing colour that would make 
readability hard for the user
	\item The application shall look professional
	\item The application shall highlight important areas
	\item The application shall grey out choices that the user cannot make
	\item The application shall have intuitive icons as well as appropriate names 
to describe the function of the button
\end{enumerate}
% End SubSubSection

\subsubsection{Style Requirements}
\label{ssub:style_requirements}
% Begin SubSubSection
\begin{enumerate}[{LF}1. ]
	\item The application shall have an appropriate style for use in the 
classroom/professional environments
\end{enumerate}
% End SubSubSection

% End SubSection

\subsection{Usability and Humanity Requirements}
\label{sub:usability_and_humanity_requirements}
% Begin SubSection

\subsubsection{Ease of Use Requirements}
\label{ssub:ease_of_use_requirements}
% Begin SubSubSection
\begin{enumerate}[{UH}1. ]
	\item The application shall make the important features stand out and easily 
accessible
	\item The application shall allow the user to get to the important information 
with no more than 2 taps of the screen
	\item The application shall have the main features on the main screen where it 
is easy to access
	\item returning to the main screen shall take no less than 2 taps
\end{enumerate}
% End SubSubSection

\subsubsection{Personalization and Internationalization Requirements}
\label{ssub:personalization_and_internationalization_requirements}
% Begin SubSubSection
\begin{enumerate}[{UH}1. ]
	\item The application shall allow the user to change the language of the 
application to English or French.
	\item The application shall allow the user to adjust what information they want 
to see on 
the front page of the application.
	\item The application shall allow the user to adjust the order of their class
\end{enumerate}
% End SubSubSection

\subsubsection{Learning Requirements}
\label{ssub:learning_requirements}
% Begin SubSubSection
\begin{enumerate}[{UH}1. ]
		\item The application shall have a tutorial to outline all of the features 
that it has to offer the user
\end{enumerate}
% End SubSubSection

\subsubsection{Understandability and Politeness Requirements}
\label{ssub:understandability_and_politeness_requirements}
% Begin SubSubSection
\begin{enumerate}[{UH}1. ]
	\item
\end{enumerate}
% End SubSubSection

\subsubsection{Accessibility Requirements}
\label{ssub:accessibility_requirements}
% Begin SubSubSection
\begin{enumerate}[{UH}1. ]
	\item
\end{enumerate}
% End SubSubSection

% End SubSection

\subsection{Performance Requirements}
\label{sub:performance_requirements}
% Begin SubSection

\subsubsection{Speed and Latency Requirements}
\label{ssub:speed_and_latency_requirements}
% Begin SubSubSection
\begin{enumerate}[{PR}1. ]
	\item The application shall take less than 2 seconds to start up.
	\item Moving in between different parts of the application shall take less than 
1 second.
\end{enumerate}
% End SubSubSection

\subsubsection{Safety-Critical Requirements}
\label{ssub:safety_critical_requirements}
% Begin SubSubSection
\begin{enumerate}[{PR}1. ]
	\item
\end{enumerate}
% End SubSubSection

\subsubsection{Precision or Accuracy Requirements}
\label{ssub:precision_or_accuracy_requirements}
% Begin SubSubSection
\begin{enumerate}[{PR}1. ]
	\item
\end{enumerate}
% End SubSubSection

\subsubsection{Reliability and Availability Requirements}
\label{ssub:reliability_and_availability_requirements}
% Begin SubSubSection
\begin{enumerate}[{PR}1. ]
	\item
\end{enumerate}
% End SubSubSection

\subsubsection{Robustness or Fault-Tolerance Requirements}
\label{ssub:robustness_or_fault_tolerance_requirements}
% Begin SubSubSection
\begin{enumerate}[{PR}1. ]
	\item
\end{enumerate}
% End SubSubSection

\subsubsection{Capacity Requirements}
\label{ssub:capacity_requirements}
% Begin SubSubSection
\begin{enumerate}[{PR}1. ]
	\item The application shall only allow 1 user/account at a time.
\end{enumerate}
% End SubSubSection

\subsubsection{Scalability or Extensibility Requirements}
\label{ssub:scalability_or_extensibility_requirements}
% Begin SubSubSection
\begin{enumerate}[{PR}1. ]
	\item
\end{enumerate}
% End SubSubSection

\subsubsection{Longevity Requirements}
\label{ssub:longevity_requirements}
% Begin SubSubSection
\begin{enumerate}[{PR}1. ]
	\item The application shall be supported and updated regularly to fix bugs and 
add new features
\end{enumerate}
% End SubSubSection

% End SubSection

\subsection{Operational and Environmental Requirements}
\label{sub:operational_and_environmental_requirements}
% Begin SubSection

\subsubsection{Expected Physical Environment}
\label{ssub:expected_physical_environment}
% Begin SubSubSection
\begin{enumerate}[{OE}1. ]
	\item
\end{enumerate}
% End SubSubSection

\subsubsection{Requirements for Interfacing with Adjacent Systems}
\label{ssub:requirements_for_interfacing_with_adjacent_systems}
% Begin SubSubSection
\begin{enumerate}[{OE}1. ]
	\item
\end{enumerate}
% End SubSubSection

\subsubsection{Productization Requirements}
\label{ssub:productization_requirements}
% Begin SubSubSection
\begin{enumerate}[{OE}1. ]
	\item
\end{enumerate}
% End SubSubSection

\subsubsection{Release Requirements}
\label{ssub:release_requirements}
% Begin SubSubSection
\begin{enumerate}[{OE}1. ]
	\item
\end{enumerate}
% End SubSubSection

% End SubSection

\subsection{Maintainability and Support Requirements}
\label{sub:maintainability_and_support_requirements}
% Begin SubSection

\subsubsection{Maintenance Requirements}
\label{ssub:maintenance_requirements}
% Begin SubSubSection
\begin{enumerate}[{MS}1. ]
	\item
\end{enumerate}
% End SubSubSection

\subsubsection{Supportability Requirements}
\label{ssub:supportability_requirements}
% Begin SubSubSection
\begin{enumerate}[{MS}1. ]
	\item
\end{enumerate}
% End SubSubSection

\subsubsection{Adaptability Requirements}
\label{ssub:adaptability_requirements}
% Begin SubSubSection
\begin{enumerate}[{MS}1. ]
	\item
\end{enumerate}
% End SubSubSection

% End SubSection

\subsection{Security Requirements}
\label{sub:security_requirements}
% Begin SubSection

\subsubsection{Access Requirements}
\label{ssub:access_requirements}
% Begin SubSubSection
\begin{enumerate}[{SR}1. ]
	\item
\end{enumerate}
% End SubSubSection

\subsubsection{Integrity Requirements}
\label{ssub:integrity_requirements}
% Begin SubSubSection
\begin{enumerate}[{SR}1. ]
	\item
\end{enumerate}
% End SubSubSection

\subsubsection{Privacy Requirements}
\label{ssub:privacy_requirements}
% Begin SubSubSection
\begin{enumerate}[{SR}1. ]
	\item
\end{enumerate}
% End SubSubSection

\subsubsection{Audit Requirements}
\label{ssub:audit_requirements}
% Begin SubSubSection
\begin{enumerate}[{SR}1. ]
	\item
\end{enumerate}
% End SubSubSection

\subsubsection{Immunity Requirements}
\label{ssub:immunity_requirements}
% Begin SubSubSection
\begin{enumerate}[{SR}1. ]
	\item
\end{enumerate}
% End SubSubSection

% End SubSection

\subsection{Cultural and Political Requirements}
\label{sub:cultural_and_political_requirements}
% Begin SubSection

\subsubsection{Cultural Requirements}
\label{ssub:cultural_requirements}
% Begin SubSubSection
\begin{enumerate}[{CP}1. ]
	\item
\end{enumerate}
% End SubSubSection

\subsubsection{Political Requirements}
\label{ssub:political_requirements}
% Begin SubSubSection
\begin{enumerate}[{CP}1. ]
	\item
\end{enumerate}
% End SubSubSection

% End SubSection

\subsection{Legal Requirements}
\label{sub:legal_requirements}
% Begin SubSection

\subsubsection{Compliance Requirements}
\label{ssub:compliance_requirements}
% Begin SubSubSection
\begin{enumerate}[{LR}1. ]
	\item
\end{enumerate}
% End SubSubSection

\subsubsection{Standards Requirements}
\label{ssub:standards_requirements}
% Begin SubSubSection
\begin{enumerate}[{LR}1. ]
	\item
\end{enumerate}
% End SubSubSection

% End SubSection

% End Section

\appendix
\section{Division of Labour}
\label{sec:division_of_labour}
% Begin Section
Include a Division of Labour sheet which indicates the contributions of each 
team member. This sheet must be signed by all team members.
% End Section

\newpage
\section*{IMPORTANT NOTES}
\begin{itemize}
	\item Be sure to include all sections of the template in your document 
regardless whether you have something to write for each or not
	\begin{itemize}
		\item If you do not have anything to write in a section, indicate this by the 
\emph{N/A}, \emph{void}, \emph{none}, etc.
	\end{itemize}
	\item Uniquely number each of your requirements for easy identification and 
cross-referencing
	\item Highlight terms that are defined in Section~1.3 (\textbf{Definitions, 
Acronyms, and Abbreviations}) with \textbf{bold}, \emph{italic} or 
\underline{underline}
	\item For Deliverable 1, please highlight, in some fashion, all (you may have 
more than one) creative and innovative features. Your creative and innovative 
features will generally be described in Section~2.2 (\textbf{Product 
Functions}), but it will depend on the type of creative or innovative features 
you are including.
\end{itemize}


\end{document}
%------------------------------------------------------------------------------

