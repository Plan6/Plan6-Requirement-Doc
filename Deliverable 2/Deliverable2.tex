\documentclass[]{article}

% Imported Packages
%------------------------------------------------------------------------------
\usepackage{amssymb}
\usepackage{amstext}
\usepackage{amsthm}
\usepackage{amsmath}
\usepackage{enumerate}
\usepackage{fancyhdr}
\usepackage[margin=1in]{geometry}
\usepackage{graphicx}
\usepackage{extarrows}
\usepackage{setspace}
%------------------------------------------------------------------------------

% Header and Footer
%------------------------------------------------------------------------------
\pagestyle{plain}
\renewcommand\headrulewidth{0.4pt}
\renewcommand\footrulewidth{0.4pt}
%------------------------------------------------------------------------------

% Title Details
%------------------------------------------------------------------------------
\title{
  Erudite\\
  \large \emph{An educational content management system}\\
  \vspace{1em}
  High-Level Architectural Design
}
\author{SE 3A04: Software Design II -- Large System Design}
\date{}
%------------------------------------------------------------------------------

% Document
%------------------------------------------------------------------------------
\begin{document}

\maketitle

\section{Introduction}
\label{sec:introduction}
This section outlines the purpose and provides a system description of the
Erudite project; along with an overview of the contents and organization of
this high-level architectural design document.


\subsection{Purpose}
\label{sub:purpose}
The purpose of this document to define the use cases, layout the Analysis Class
Diagram, describe the Architectural Design, and finally document the class
responsibilities through collaboration cards. This document builds on top of
and extends the Software Requirements Specification document in that this
document describes the way in which the system will interact with the outside
world and how the subsystems will be architecturally and logically arranged.

The target audience for this document are the stakeholders (Dr. Ridha Khedri,
Andrew Le Clair and Michael Liut), and any current or future architects,
designers and developers of this project.


\subsection{System Description}
\label{sub:system_description}
The Erudite application is intended to be a educational content management
system for use in elementary school classrooms. The primary interface between
the user and the software system is through a device running the Android
operating system. This document defines the way in which the users will be
expected to interact with the system and how the application will be decomposed
into smaller subsystems to reduce the complexity and improve the
maintainability, flexibility of this system.

Specifically, the users of this system (application) are expected to perform a
set of events that will prompt the system to react. The decomposition will then
show how the subsystems will communicate among one another in order to
efficiently distribute the work and perform the required actions in response to
the user's actions.

\subsection{Overview}
\label{sub:overview}
The remainder of the document is organized into 4 sections: Use Case Diagram --
how the users and system will interact, Analysis Class Diagram -- the
subsystems that compose this entire application, Architectural Design -- the
layout of the subsystems into a well-understood software architecture, and
Class responsibility collaboration Cards -- description of the interactions
between the subsystems. Each section uses an appropriate notation and diagrams
to document the design decision and describe the details of the high-level
design of this system.


% End Section


\section{Use Case Diagram}
\label{sec:use_case_diagram}
% Begin Section
This section should provide a use case diagram for your application. 
\begin{enumerate}[a)]
	\item Each use case appearing in the diagram should be accompanied by a text description. 
\end{enumerate}
% End Section

\section{Analysis Class Diagram}
\label{sec:analysis_class_diagram}
% Begin Section
This section should provide an analysis class diagram for your application.
% End Section


\section{Architectural Design}
\label{sec:architectural_design}
% Begin Section
This section should provide an overview of the overall architectural design of your application. You overall architecture should show the division of the system into subsystems with high cohesion and low coupling.

\subsection{System Architecture}
\label{sub:system_architecture}
% Begin SubSection
\begin{enumerate}[a)]
	\item Identify and explain the overall architecture of your system
	\item Be sure to clearly state the name of the architecture
	\item Provide the reasoning and justification of the choice
	\item Provide a structural architecture diagram showing the relationship among the subsystems (if appropriate)
\end{enumerate}
% End SubSection

\subsection{Subsystems}
\label{sub:subsystems}
% Begin SubSection
\begin{enumerate}[a)]
	\item Provide a brief description of each subsystem. Be sure to document its purpose and relationship to other subsystems.
\end{enumerate}
% End SubSection

% End Section
	
\section{Class Responsibility Collaboration (CRC) Cards}
\label{sec:class_responsibility_collaboration_crc_cards}
% Begin Section
This section should contain all of your CRC cards.

\begin{enumerate}[a)]
	\item Provide a CRC Card for each identified class
	\item Please use the format outlined in tutorial, i.e., 
	\begin{table}[ht]
		\centering
		\begin{tabular}{|p{5cm}|p{5cm}|}
		\hline 
		 \multicolumn{2}{|l|}{\textbf{Class Name:}} \\
		\hline
		\textbf{Responsibility:} & \textbf{Collaborators:} \\
		\hline
		\vspace{1in} & \\
		\hline
		\end{tabular}
	\end{table}
	
\end{enumerate}
% End Section

\newpage
\appendix
\section{Division of Labour}
\label{sec:division_of_labour}
\begin{description}
  \item [Kelvin Lin ]
  \item{Foo}
  \hfill \rule{2in}{0.1pt}
  \\\\

  \item [Danish Khan]
  \item{Bar}
  \hfill \rule{2in}{0.1pt}
  \\\\

  \item [Puru Jetly]
  \item{Baz}
  \hfill \rule{2in}{0.1pt}
  \\\\

  \item [Terrance Yip]
  \item{Qux}
  \hfill \rule{2in}{0.1pt}
  \\\\

  \item [Varun Hooda]
  \item{Section 1 Introduction}
  \item{Revision to use case diagram}
  \hfill \rule{2in}{0.1pt}
  \\\\
\end{description}

\newpage
\section*{IMPORTANT NOTES}
\begin{itemize}
%	\item You do \underline{NOT} need to provide a text explanation of each diagram; the diagram should speak for itself
	\item Please document any non-standard notations that you may have used
	\begin{itemize}
		\item \emph{Rule of Thumb}: if you feel there is any doubt surrounding the meaning of your notations, document them
	\end{itemize}
	\item Some diagrams may be difficult to fit into one page
	\begin{itemize}
		\item It is OK if the text is small but please ensure that it is readable when printed
		\item If you need to break a diagram onto multiple pages, please adopt a system of doing so and thoroughly explain how it can be reconnected from one page to the next; if you are unsure about this, please ask about it
	\end{itemize}
	\item Please submit the latest version of Deliverable 1 with Deliverable 2
	\begin{itemize}
		\item It does not have to be a freshly printed version; the latest marked version is OK
	\end{itemize}
	\item If you do \underline{NOT} have a Division of Labour sheet, your deliverable will \underline{NOT} be marked
\end{itemize}


\end{document}
%------------------------------------------------------------------------------
